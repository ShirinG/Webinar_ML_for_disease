\documentclass[notes, c, 11pt, xcolor=svgnames, hyperref={colorlinks,citecolor=DeepPink4,linkcolor=DarkRed,urlcolor=DarkBlue}]{beamer}

% print notes only:
%\documentclass[notes=only, c, 11pt, xcolor=svgnames, hyperref={colorlinks,citecolor=DeepPink4,linkcolor=DarkRed,urlcolor=DarkBlue}]{beamer}

% print only frames:
%\documentclass[c, 11pt, xcolor=svgnames, hyperref={colorlinks,citecolor=DeepPink4,linkcolor=DarkRed,urlcolor=DarkBlue}]{beamer}

% print handouts:
%\documentclass[handout, c, 11pt, xcolor=svgnames, hyperref={colorlinks,citecolor=DeepPink4,linkcolor=DarkRed,urlcolor=DarkBlue}]{beamer} 
%\usepackage{pgfpages}
%\pgfpagesuselayout{4 on 1}[a4paper, border shrink=5mm, landscape]

\setbeamerfont{note page}{size=\tiny}

\usepackage[english, german]{babel}
\usepackage[latin1]{inputenc}
\usepackage[T1]{fontenc}

\mode<presentation> {
	
	% The Beamer class comes with a number of default slide themes
	% which change the colors and layouts of slides. Below this is a list
	% of all the themes, uncomment each in turn to see what they look like.
	
	%\usetheme{default}
	%\usetheme{AnnArbor}
	%\usetheme{Antibes}
	%\usetheme{Bergen}
	%\usetheme{Berkeley}
	%\usetheme{Berlin}
	%\usetheme{Boadilla}
	\usetheme{CambridgeUS}
	%\usetheme{Copenhagen}
	%\usetheme{Darmstadt}
	%\usetheme{Dresden}
	%\usetheme{Frankfurt}
	%\usetheme{Goettingen}
	%\usetheme{Hannover}
	%\usetheme{Ilmenau}
	%\usetheme{JuanLesPins}
	%\usetheme{Luebeck}
	%\usetheme{Madrid}
	%\usetheme{Malmoe}
	%\usetheme{Marburg}
	%\usetheme{Montpellier}
	%\usetheme{PaloAlto}
	%\usetheme{Pittsburgh}
	%\usetheme{Rochester}
	%\usetheme{Singapore}
	%\usetheme{Szeged}
	%\usetheme{Warsaw}
	
	% As well as themes, the Beamer class has a number of color themes
	% for any slide theme. Uncomment each of these in turn to see how it
	% changes the colors of your current slide theme.
	
	%\usecolortheme{albatross}
	%\usecolortheme{beaver}
	%\usecolortheme{beetle}
	%\usecolortheme{crane}
	%\usecolortheme{dolphin}
	%\usecolortheme{dove}
	%\usecolortheme{fly}
	%\usecolortheme{lily}
	%\usecolortheme{orchid}
	%\usecolortheme{rose}
	%\usecolortheme{seagull}
	%\usecolortheme{seahorse}
	%\usecolortheme{whale}
	%\usecolortheme{wolverine}
	
	\usefonttheme{professionalfonts}
	
	\usepackage{times}
	\usepackage{tikz}
	\usetikzlibrary{arrows,shapes}
	
	%\setbeamertemplate{footline} % To remove the footer line in all slides uncomment this line
	%\setbeamertemplate{footline}[page number] % To replace the footer line in all slides with a simple slide count uncomment this line
	\setbeamertemplate{headline} % to remove header with navigation tree
	
	\setbeamertemplate{navigation symbols}{} % To remove the navigation symbols from the bottom of all slides uncomment this line
}

\usepackage{amsmath,mathtools}
\usetikzlibrary{matrix, fit, backgrounds}

\usepackage{graphicx} % Allows including images
\usepackage{booktabs} % Allows the use of \toprule, \midrule and \bottomrule in tables
\usepackage{subcaption}
\usepackage{tabularx}

\usetikzlibrary{mindmap,trees,shadows,backgrounds}

%\tikzset{every node/.append style={scale=0.6}}    

\usepackage{array,multirow}
\usepackage{changepage}

\usepackage[backend=biber, style=authoryear-comp, citestyle=authoryear-comp, firstinits=true, url=false, doi=false, eprint=false, dashed=false, maxbibnames=99]{biblatex}
\addbibresource{bib.bib}

%----------------------------------------------------------------------------------------
%	TITLE PAGE
%----------------------------------------------------------------------------------------

\title[Webinar for ISDS R Group]{Building meaningful machine learning models for disease prediction} % The short title appears at the bottom of every slide, the full title is only on the title page

\author{Dr Shirin Glander} % Your name
\institute[] % Your institution as it will appear on the bottom of every slide, may be shorthand to save space
{
	Dep. of Genetic Epidemiology \\
	Institute of Human Genetics \\
	University of M\"unster
	
	\vspace{0.5cm} 
	
	\href{mailto:shirin.glander@wwu.de}{shirin.glander@wwu.de} \\
	
	\vspace{0.5cm} 
	
	\href{https://shiring.github.io}{https://shiring.github.io} \\
	
	\href{https://github.com/ShirinG}{https://github.com/ShirinG}
	
}
\date{31.03.2017
	\vspace{0.5cm}
} % Date, can be changed to a custom date

\AtBeginSection[]{
	\begin{frame}
		\vfill
		\centering
		\begin{beamercolorbox}[sep=8pt,center,shadow=true,rounded=true]{title}
			\usebeamerfont{title}\insertsectionhead\par%
		\end{beamercolorbox}
		\vfill
	\end{frame}
}

%\renewcommand{\thefootnote}{$\star$} 

\begin{document}
	
	\begin{frame}
		\titlepage % Print the title page as the first slide
		
		%Dr Shirin Glander will go over her work on building machine-learning models to predict the course of different diseases. She will go over building a model, evaluating its performance, and answering or addressing different disease related questions using machine learning. Her talk will cover the theory of machine learning as it is applied using R. 
	\end{frame}
	
	\begin{frame}[c]
		\frametitle{Table of contents}
		
			\tableofcontents[hideallsubsections]

	\end{frame}

\begin{frame}
	\frametitle{About me}
	
	\begin{columns}
		\column{0.15\textwidth}	
	
		\column{0.8\textwidth}	
	\begin{itemize}
		\item[2005 - 2011] BSc and MSc of Science in Biology \\
							Evolutionary genetics, \\ immune memory in Drosophila \\[5mm]
		\item[2011 - 2015] PhD in Biology \\
							Is the immune system of plants required to adapt to flowering time change? \\[5mm]
		\item[since 2015] Bioinformatics Postdoc \\
							Autoinflammatory diseases \& innate immunity \\
							Next Generation Sequencing
	\end{itemize}

	\column{0.05\textwidth}	
	
	\begin{tikzpicture}[remember picture,overlay]
	\node[xshift=-2cm,yshift=-2.3cm] at (current page.north east) {\includegraphics[width=3cm]{images/Bewerbungsfoto}};
	\end{tikzpicture}

	\end{columns}
\end{frame}

%%%%%%%%%%%%%%%%%%%%%

\section{What makes a model meaningful?}

\note{
	Machine learning is a powerful approach for developing sophisticated, automatic, and objective algorithms for analysis of high-dimensional and multimodal biomedical data. \\
	
	\vspace{0.2cm}
	A key aspect of the precision medicine effort is the development of informatics tools that can analyze and interpret 'big data' sets in an automated and adaptive fashion while providing accurate and actionable clinical information. \\
	
	\vspace{0.1cm}
	ML based model can improve detection, diagnosis, and therapeutic monitoring of disease. \\
}

\begin{frame}
	\frametitle{\textsl{Meaningful} models}
	
	\begin{itemize}
		\item answer the question(s) posed...
		\item ... with sufficient accuracy to be trustworthy
	\end{itemize}

\vspace{0.6cm}

	\begin{center}
		{\Large \alert{Accuracy depends on the problem!}}
		
		\vspace{1cm}
		
		\includegraphics[width=0.33\textwidth]{images/meaningful_1}
		\includegraphics[width=0.33\textwidth]{images/meaningful_2}
		\includegraphics[width=0.33\textwidth]{images/meaningful_3}
	\end{center}
	
\end{frame}

\note{
	I want to begin with an introduction to the main question of my talk: what makes a model \textsl{good} or \textsl{meaningful}? \\
	\vspace{0.1cm}
	
	A meaningful model
	\begin{itemize}
		\item answers the questions or questions posed...
		\item ... with sufficient accuracy to be trustworthy
	\end{itemize}

	\vspace{0.1cm}
	But what we mean exactly with \textsl{accuracy} can not be defined with a one-size-fits-all approach: it depends on the problem we want to model. 	

	\vspace{0.1cm}
	Let me illustrate what I mean with the following examples:
	\begin{itemize}
		\item \textbf{Ideal case:} Of course, we all want to achieve ideal modeling results where overall prediction accuracy is very high. \\ With a model like that, we can be very confident that a healthy person is indeed healthy and a sick person is not. \\
		\item But in reality, we often achieve prediction accuracies that are much less nice. \\
		\item This forces us to evaluate how we want to define a good model
		\item \alert{Scenarios 1 and 2:} Le's consider two possible scenarios: 
		\item in case 1, we can be very confident that a person who got classified as "healthy" is indeed healthy, \\ while a person who has been diagnosed as diseased might as well be healthy based on these prediction accuracies
		\item in case 2, it is the other way around.
		\item We now need to make a decision which scenario is better and in which direction we want to optimize our model: \\ do we rather want to refer a few healthy people for further checking because the model predicted them as diseased? \\ Or do we rather want to be as certain as possible that a predicted disease is actually true \\ and accept that we might miss a few disease cases?
	\end{itemize}
}
%%%%%%%%%%%%%%%%%%%%%

\section{Machine Learning (ML) in disease modeling}

%publications

\begin{frame}
	\frametitle{ML in disease modeling}
	
	\begin{itemize}
		\item falls into the field of \alert{artificial intelligence (AI)}
		\item algorithms \alert{learn} by being trained on observed data
		\item learned models can \alert{predict unknown data}
		\item ML concepts are not new, but the increase in computational capacity has made them more accessible
	\end{itemize}

\vspace{0.2cm}

\alert{Examples:} 
\begin{itemize}
	\item Computer-aided diagnosis of breast cancer from mammograms,
	\item early diagnosis of osteoporosis from chest radiographs, etc. \footfullcite{Kunio}
	\item Identifying signatures of Brain Cancer from MRSI \footfullcite{doi:10.1146/annurev.bioeng.8.061505.095802}
\end{itemize}
	
	%\footcite{}
	
\end{frame}

\note{
	Computer-aided diagnosis (CAD) has become one of the major research subjects in medical imaging and diagnostic radiology \\
	With CAD, radiologists use the computer output as a 'second opinion' and make the final decisions. \\
	
	\vspace{0.2cm}
	In vivo magnetic resonance spectroscopy imaging (MRSI) allows noninvasive characterization and quantification of molecular markers of potentially high clinical utility for improving detection, identification, and treatment for a variety of diseases, most notably brain cancers. \\
	
	\vspace{0.2cm}
	1. Unsupervised matrix decomposition methods, such as nonnegative matrix
	factorization, which impose general, although physically meaningful, constraints,
	are able to recover biomarkers of disease and toxicity, generating a
	natural basis for data visualization and pattern classification. \\
	
	\vspace{0.2cm}
	2. Supervised discriminative models that explicitly address the bias-variance
	trade-off, such as the support vector machine, have shown great promise for
	disease diagnosis in computational biology, where data types are disparate
	and high dimensional. \\
	
	\vspace{0.2cm}
	3. Generative models based on Bayesian networks offer a general approach
	for biomedical image and signal analysis in that they enable one to directly
	model the uncertainty and variability inherent to biomedical data as well
	as provide a framework for an array of analysis, including classification,
	segmentation, and compression. \\
}

%%%%%%%%%%%%%%%%%%%%%

\section{A quick recap of ML basics}

\begin{frame}
	\frametitle{Supervised vs Unsupervised algorithms}

\end{frame}


\begin{frame}
	\frametitle{Classification vs Regression}

\end{frame}


\begin{frame}
	\frametitle{Features}
	
	\begin{itemize}
		\item feature selection
		\item feature extraction
		\item feature engineering
	\end{itemize}
\end{frame}

% factorial vs numeric
% preprocessing

\note{
	extraction of salient structure in the data that is more informative than the raw data itself (the feature extraction problem)
}

\subsection{Training, validation and test data}

% preventing overfitting, we want our models to be generalizeable

\subsection{Cross-validation}

\subsection{Missing data}

\subsection{Grid Search}

%imputation


%%%%%%%%%%%%%%%%%%%%%

\section{How to build ML models in R}

\begin{frame}
	\frametitle{Session setup}
	
	Code will be available on \href{https://shiring.github.io}{my website} and on \href{https://github.com/ShirinG/Webinar_ML_for_disease}{Github}
	
	Breast cancer Wisconsin dataset
	
	caret package
	
	h2o package
\end{frame}

\subsection{Get to know your data}

\begin{frame}
	\frametitle{Distribution}
	
	\begin{center}\includegraphics[width=1\textwidth]{webinar_code_files/figure-latex/unnamed-chunk-6-1} \end{center}
\end{frame}


\subsection{Classification}

\subsubsection{Tree based models}

\begin{frame}
	\frametitle{Decision trees}
	
	\begin{center}\includegraphics[width=1\textwidth]{webinar_code_files/figure-latex/unnamed-chunk-12-1} \end{center}
	
\end{frame}

\begin{frame}
	\frametitle{Random Forest}
	
	
\end{frame}

\begin{frame}
	\frametitle{Feature importance}
	
	\begin{center}\includegraphics[width=1\textwidth]{webinar_code_files/figure-latex/unnamed-chunk-11-1.pdf} \end{center}
\end{frame}

\subsection{Regression}

\subsubsection{Linear Models}

\begin{frame}
	\frametitle{(Generalized) Linear Models}
	
	\begin{center}\includegraphics[width=1\textwidth]{webinar_code_files/figure-latex/unnamed-chunk-21-1.pdf} \end{center}
\end{frame}

\subsubsection{Deep learning with neural network}

% h2o


%%%%%%%%%%%%%%%%%%%%%

\section{Evaluating ML model performance}

\subsection{Validation vs test set performance}

\begin{frame}
	\frametitle{AUC and MSE}
	
	\begin{center}\includegraphics[width=1\textwidth]{webinar_code_files/figure-latex/unnamed-chunk-29-1.pdf} \end{center}
	
	\begin{center}\includegraphics[width=1\textwidth]{webinar_code_files/figure-latex/unnamed-chunk-42-1} \end{center}
\end{frame}

\begin{frame}
	\frametitle{Predictions on test data}
	
	\begin{center}\includegraphics[width=1\textwidth]{webinar_code_files/figure-latex/unnamed-chunk-31-1.pdf}
	
	\includegraphics[width=1\textwidth]{webinar_code_files/figure-latex/unnamed-chunk-31-2.pdf} \end{center}
\end{frame}

% accuracy
% Kappa


%%%%%%%%%%%%%%%%%%%%%

\begin{frame}[plain, c]
	
	\begin{center}
	\usebeamerfont*{frametitle} \usebeamercolor[fg]{frametitle} {\Huge \textbf{Thank you for your attention!}}
	\end{center}

	\vspace{0.5cm}

	\begin{center}
		\usebeamerfont*{frametitle} {\huge Questions?}
	\end{center}

	\vspace{0.5cm}
	
	Slides and code will be available on Github: \href{https://github.com/ShirinG/Webinar_ML_for_disease}{https://github.com/ShirinG/Webinar\_ML\_for\_disease} \\
	
	\vspace{0.5cm}
	
	Code will also be on my website: \href{https://shiring.github.io}{https://shiring.github.io} \\
	
	\vspace{0.5cm}
	
	\href{mailto:shirin.glander@wwu.de}{shirin.glander@wwu.de} \\
	
\end{frame}

\end{document}